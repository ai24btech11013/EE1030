%iffalse
\let\negmedspace\undefined
\let\negthickspace\undefined
\documentclass[journal,12pt,onecolumn]{IEEEtran}
\usepackage{cite}
\usepackage{amsmath,amssymb,amsfonts,amsthm}
\usepackage{algorithmic}
\usepackage{graphicx}
\usepackage{textcomp}
\usepackage{xcolor}
\usepackage{txfonts}
\usepackage{listings}
\usepackage{enumitem}
\usepackage{mathtools}
\usepackage{gensymb}
\usepackage{comment}
\usepackage[breaklinks=true]{hyperref}
\usepackage{tkz-euclide} 
\usepackage{listings}
\usepackage{gvv}                                        
%\def\inputGnumericTable{}                                 
\usepackage[latin1]{inputenc}                                
\usepackage{color}                                            
\usepackage{array}                                            
\usepackage{longtable}                                       
\usepackage{calc}                                             
\usepackage{multirow}                                         
\usepackage{hhline}                                           
\usepackage{ifthen}                                           
\usepackage{lscape}
\usepackage{tabularx}
\usepackage{array}
\usepackage{float}
\newtheorem{theorem}{Theorem}[section]
\newtheorem{problem}{Problem}
\newtheorem{proposition}{Proposition}[section]
\newtheorem{lemma}{Lemma}[section]
\newtheorem{corollary}[theorem]{Corollary}
\newtheorem{example}{Example}[section]
\newtheorem{definition}[problem]{Definition}
\newcommand{\BEQA}{\begin{eqnarray}}
\newcommand{\EEQA}{\end{eqnarray}}
\newcommand{\define}{\stackrel{\triangle}{=}}
\theoremstyle{remark}
\newtheorem{rem}{Remark}

% Marks the beginning of the document
\begin{document}
\bibliographystyle{IEEEtran}
\vspace{3cm}

\title{Jee 2022 shift-1 16-30}
\author{AI24BTECH11013-Geetha charani}
\maketitle
\bigskip

\renewcommand{\thefigure}{\theenumi}
\renewcommand{\thetable}{\theenumi}
\section{Section - A}
\begin{enumerate}
    \item The area of the region ${\brak{x, y} : |x-1| \leq y \leq \sqrt{5-x^2}}$ is equal to:
\begin{enumerate}
    \item $\frac{5}{2}\sin^{-1}(\frac{3}{5})-\frac{1}{2}$
    \item $\frac{5\pi}{4}-\frac{3}{2}$
    \item $\frac{3\pi}{4}+\frac{3}{2}$
    \item $\frac{5\pi}{4}-\frac{1}{2}$
\end{enumerate}
\item Let the focal chord of the parabola P : $y^2 = 4x$ along the line L : $y = mx + c, m>0$ meet the parabola at the points M and N. Let the line L be the tangent to the hyperbola H : $x^2 + y^2 = 4$. If O is the vertex of P and F is the focus of H on the positive x-axis, then the area of the quadrilateral OMFN is :
\begin{enumerate}
    \item $2\sqrt{6}$
    \item $2\sqrt{14}$
    \item $4\sqrt{6}$
    \item $4\sqrt{14}$
\end{enumerate}
\item The number of points, where the function $f : R \rightarrow R, f(x) = |x - 1| cos|x-2| sin|x -1| + \brak{x-3}|x^2 - 5x + 4|$, is NOT differentiable, is  :
\begin{enumerate}
    \item 1
    \item 2
    \item 3
    \item 4
\end{enumerate}
\item Let S = {1, 2, 3, ....,2022}. Then the probability, that a randomly chosen number n from the set S such that HCF\brak{n,2022} = 1, is :
\begin{enumerate}
    \item $\frac{128}{1011}$
    \item $\frac{166}{1011}$
    \item $\frac{127}{1011}$
    \item $\frac{112}{1011}$
\end{enumerate}
\item Let $f\brak{x} = 3^{\brak{x^2 - 2}^3+4}, x \in R$. Then which of the following statements are true ?\\
P : $x = 0$ is a point of local minima of f\\
Q : $x +\sqrt{2}$ is a point of inflection of f\\
R : $f^{\prime}$ is increasing for $x > \sqrt{2}$
\begin{enumerate}
    \item Only P and Q
    \item Only P and R
    \item Only Q and R
    \item All P, Q and R
\end{enumerate}
\section{Section - B}
\item Let $S = \theta \in \brak{0,2\pi} : 7 cos^2\theta - 3 sin^2 \theta - 2 cos^2\theta = 2$. Then the sum of roots of all the equations $x^2 - 2 \brak{tan^2\theta + cot^2\theta} x + 6 sin^2\theta = 0 $ $\theta \in S$, is :
\item Let the mean and the variance of 20 observations $x_1,x_2,....x_20$ be 15 and 9, respectively. For $\alpha \in R$, if the mean of $\brak{x_1 + \alpha}^2,\brak{x_2 + \alpha}^2,...brak{x_20 + \alpha}^2$ is 178, then the square of the maximum value of $\alpha$ is eqal to :
\item let a line with direction ratios a, -4a, -7 be perpendicular to the lines with direction ratios 3, -1, 2b and b, a, -2. If the point of intersection of the line $\frac{x + 1}{a^2 +b^2} = \frac{y - 2}{a^2 - b^2} = \frac{z}{1}$ and the plane $x - y + z = 0$ is $\brak{\alpha, \beta, \gamma}$, then $\alpha + \beta + \gamma$ is equal to 
\item Let $a_1, a_2, a_3,....$ be an A.P. If $\sum_{n=1}^{\infty} \frac{a_r}{2^r}=4$, then $4a_2$ is equal to 
\item Let the ratio of the fifth term from the beginning to the fifth term from the end in the binomial expansion of $\brak{\sqrt[4]{2}+\frac{1}{\sqrt[4]{3}}}^n$, in the increasing powers of $\frac{1}{\sqrt[4]{3}}$ be $\sqrt[4]{6} : 1$. If the sixth term from the beginning is $\frac{\alpha}{\sqrt[4]{3}}$, then $\alpha$ is equal to 
\item Let number of matrices of order 3 * 3, whose entries are either 0 or 1 and the sum of all the entries is a prime number, is 
\item Let p and p + 2 be prime numbers and let \\
$\Delta = 
\begin{vmatrix}
p! & \brak{p+1}! & \brak{p+2}! \\
\brak{p+1}! & \brak{p+2}! & \brak{p+3}! \\
\brak{p+2}! & \brak{p+3}! & \brak{p+4}!
\end{vmatrix}$

Then the sum of the maximum values of $\alpha$ and $\beta$, such that $p^\alpha$ and $\brak{p +2}^\beta$ divide $\Delta$, is 
\item If $\frac{1}{2*3*4} + \frac{1}{3*4*5} + \frac{1}{4*5*6} +....+ \frac{1}{100*101*102} = \frac{k}{101}$, then 34 k is equal to
\item Let $S = {4,6,9}$ and $T = {9,10,11,....1000}$. If $A = {a_1+a_2+...+a_k : K\in N, a_1,a_2,a_3,...,a_k \in S}$, then the sum of all the elements in the set $T - A$ is equal to 
\item Let the mirrir image of a circle $c_1 : x^2 +y^2 - 2x - 6y +\alpha = 0$ in line $y = x + 1$ be $c_2 : 5x^2 + 5y^2 +10gx + 10fy + 38 =  0.$ If r is the radius of the circle $c_2$, then $\alpha + 6r^2$ is equal to
\end{enumerate}
\end{document}
