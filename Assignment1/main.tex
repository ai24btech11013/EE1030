%iffalse
\let\negmedspace\undefined
\let\negthickspace\undefined
\documentclass[journal,12pt,twocolumn]{IEEEtran}
\usepackage{cite}
\usepackage{amsmath,amssymb,amsfonts,amsthm}
\usepackage{algorithmic}
\usepackage{graphicx}
\usepackage{textcomp}
\usepackage{xcolor}
\usepackage{txfonts}
\usepackage{listings}
\usepackage{enumitem}
\usepackage{mathtools}
\usepackage{gensymb}
\usepackage{comment}
\usepackage[breaklinks=true]{hyperref}
\usepackage{tkz-euclide} 
\usepackage{listings}
\usepackage{gvv}                                        
%\def\inputGnumericTable{}                                 
\usepackage[latin1]{inputenc}                                
\usepackage{color}                                            
\usepackage{array}                                            
\usepackage{longtable}                                       
\usepackage{calc}                                             
\usepackage{multirow}                                         
\usepackage{hhline}                                           
\usepackage{ifthen}                                           
\usepackage{lscape}
\usepackage{tabularx}
\usepackage{array}
\usepackage{float}
\newtheorem{theorem}{Theorem}[section]
\newtheorem{problem}{Problem}
\newtheorem{proposition}{Proposition}[section]
\newtheorem{lemma}{Lemma}[section]
\newtheorem{corollary}[theorem]{Corollary}
\newtheorem{example}{Example}[section]
\newtheorem{definition}[problem]{Definition}
\newcommand{\BEQA}{\begin{eqnarray}}
\newcommand{\EEQA}{\end{eqnarray}}
\newcommand{\define}{\stackrel{\triangle}{=}}
\theoremstyle{remark}
\newtheorem{rem}{Remark}

% Marks the beginning of the document
\begin{document}
\bibliographystyle{IEEEtran}
\vspace{3cm}

\title{Definite Integrals and Applications of Integrals}
\author{AI24BTECH11013-Geetha charani}
\maketitle
\newpage
\bigskip

\renewcommand{\thefigure}{\theenumi}
\renewcommand{\thetable}{\theenumi}
\section{Comprehension Based Questions}
\textbf{PASSAGE-1}

Let the definite integral be defined by the formula
$\int_a^b f(x) \, dx = \frac{b-a}{2} ( f(a) + f(b) ).$
For a more accurate result for \( c \in (a, b) \), we can use
$\int_a^b f(x) \, dx = \int_a^c f(x) \, dx + \int_c^b f(x) \, dx = F(c)$
so that for \( c = \frac{a+b}{2} \), we get
$\int_a^b f(x) \, dx = \frac{b-a}{4}( f(a) + f(b) + 2f(c) )$.
\begin{enumerate}
\item $\int_0^{\frac{\pi}{2}} \sin x \, dx =$
\hfill{\textbf{(2006- 5M -2)}}
\begin{enumerate}[label=(\alph*)]
    \item $ \frac{\pi}{8} ( 1 + \sqrt{2} ) $
    \item $\frac{\pi}{4} ( 1 + \sqrt{2} ) $
    \item $ \frac{\pi}{8\sqrt{2}}$
    \item  $\frac{\pi}{4\sqrt{2}}$ 
\end{enumerate}
\item $\lim_{x \to a} \frac{\int_a^x f(t)dt - (\frac{x-a}{2})(f(x) + f(a))}{(x - a)^3} = 0 $, then  f(x)  is 
 of maximum degree 
 \hfill{\textbf{(2006- 5M -2)}}
 \begin{enumerate}[label=(\alph*)]
    \item  4 
    \item  3 
    \item  2 
    \item  1
  \end{enumerate}
 \item If $ f''(x) < 0 \ \forall \ x \in (a, b) $ and  c  is a point such that $a < c < b $, and $(c, f(c))$ is the point lying on the curve for which $F(c) $ is maximum, then $f'(c) $ is equal to
 \hfill{\textbf{(2006- 5M -2)}}
 \begin{enumerate}[label=(\alph*)]
    \item  $\frac{f(b) - f(a)}{b - a} $
    \item  $ \frac{2(f(b) - f(a))}{b - a} $
    \item  $ \frac{2f(b) - f(a)}{2b - a} $
    \item  0
\end{enumerate}
\textbf{PASSAGE - 2}
 
 Consider the functions defined implicitly by the equation 
$y^3 - 3y + x = 0$
on various intervals in the real line. If $ x \in (-\infty, -2) \cup (2, \infty) $, the equation implicitly defines a unique real-valued differentiable function $ y = f(x) $. If $x \in (-2, 2)$, the equation implicitly defines a unique real valued differentiable function $y = g(x)$ satisfying $ g(0) = 0$.

\item If $f(-10\sqrt{2})=2\sqrt{2}$, then $f"(-10\sqrt{2})$=
\hfill{\textbf{2008}}
\begin{enumerate}
    \item $\frac{4\sqrt{2}}{7^3 3^2}$
    \item $\frac{-4\sqrt{2}}{7^3 3^2}$ 
     \item $\frac{4\sqrt{2}}{7^3 3}$  
     \item $\frac{-4\sqrt{2}}{7^3 3}$ 
\end{enumerate}
\item The area of the region bounded by the curve $y=f(x)$, the x-axis, and the lines x = a and x = b, where $-\infty <a<b<-2$, is
\hfill{\textbf{2008}}
\begin{enumerate}
    \item $\int_a^b\frac{x}{3((f(x))^2-1)}dx$ +$bf(b)$-$af(a)$ 
    \item -$\int_a^b\frac{x}{3((f(x))^2-1)}dx$ +$bf(b)$-$af(a)$
    \item $\int_a^b\frac{x}{3((f(x))^2-1)}dx$ -$bf(b)$+$af(a)$ 
    \item -$\int_a^b\frac{x}{3((f(x))^2-1)}dx$ -$bf(b)$+$af(a)$ 
\end{enumerate}
\item $\int_ {-1}^{1} g'(x)dx$ =
\hfill{\textbf{2008}}
\begin{enumerate}
    \item $2g(-1)$
    \item 0
    \item $-2g(1)$
    \item $2g(1)$ 
\end{enumerate}
\textbf{PASSAGE-3}
Consider the function f: $(-\infty,\infty)\rightarrow(-\infty,\infty)$ defined by $f(x)=\frac{x^2 -ax+1}{x^2+ax+1}$, $ 0<a<2.$

 \item Which of the following is true?
\hfill{\textbf{2008}}
\begin{enumerate}
    \item $(2+a)^2 f"(1)+(2-a)^2 f"(1)$ 
    \item $(2+a)^2 f"(1)-(2-a)^2 f"(1)$
    \item $f'(1)f'(-1)=(2-a)^2$ 
     \item $f'(1)f'(-1)=-(2+a)^2$ 
\end{enumerate}
\item Which of the following is true?
\hfill{\textbf{2008}}
\begin{enumerate}
    \item $f(x)$ is decreasing on $(-1,1)$ and has a local minimum at x=1
    \item $f(x)$ is increasing on $(-1,1)$ and has a local minimum at x=1
    \item $f(x)$  is increasing on $(-1,1)$ but has neither a local maximum nor a local minimum at x=1
    \item $f(x)$  is decreasing on $(-1,1)$ but has neither a local maximum nor a local minimum at x=1
\end{enumerate}
\item Let $g(x)=\int_{0}^{e^x}\frac{f'(t)}{1+t^2}dt$. Which of the following is true?
\begin{enumerate}
    \item $g'(x)$ is positive on $(-\infty,0)$ and negative on $(0,\infty)$
     \item $g'(x)$ is negative on $(-\infty,0)$ and positive on $(0,\infty)$
     \item $g'(x)$ changes sign on both $(-\infty,0)$ and $(0,\infty)$
     \item $g'(x)$ does not change sign on $(-\infty,\infty)$  
\end{enumerate}
\textbf{PASSAGE-4}

Consider the polynomial $f(x) = 1+2x+3x^2+4x^3$. Let s be the sum of all distinct real roots of $f(x)$ and let $t=|s|$.
\hfill{\textbf{2010}}

\item The real numbers lies in the interval
\begin{enumerate}
    \item $(-\frac{1}{4},0)$
    \item $(-11,-\frac{3}{4})$
    \item $(-\frac{3}{4},-\frac{1}{2})$
     \item $(0,\frac{1}{4})$
\end{enumerate}
\item The area bounded by the curve $y=f(x)$ and the lines $x=0$, $y=0$ 
 and $x=t$, lies in the interval
\begin{enumerate}
\item $(\frac{3}{4},3)$ 
\item $(\frac{21}{64},\frac{11}{16})$ 
\item $(9,10)$  
\item $(0,\frac{21}{64})$ 
\end{enumerate}
\item The function f'(x) is
\begin{enumerate}
    \item increasing in $(-t,-\frac{1}{4})$ and decreasing in $(-\frac{1}{4},t)$ 
    \item decreasing in $(-t,-\frac{1}{4})$ and increasing in $(-\frac{1}{4},t)$ 
    \item increasing in $(-t,t)$ 
    \item decreasing in $(-t,t)$
\end{enumerate}
\textbf{PASSAGE-5}

Given that for each $a \in (0, 1)$,

$\lim_{h \to 0^+} \int_h^{1-h} t^{-a} (1-t)^{a-1} \, dt \text{ exists. Let this limit be } g(a).$

In addition, it is given that the function \(g(a)\) is differentiable on $(0, 1)$. 
\hfill\textbf{JEE Adv.2014} 
    \item The value of $g(\frac{1}{2})$ is
    \begin{enumerate}
        \item $\pi$
        \item $2\pi$
        \item $\frac{\pi}{2}$
        \item $\frac{\pi}{4}$
    \end{enumerate}
    \item The value of $g'(\frac{1}{2})$ is
    \begin{enumerate}
        \item$\frac{\pi}{2}$
        \item${\pi} $
        \item $-\frac{\pi}{2}$
        \item 0
\end{enumerate}
\end{enumerate}
\end{document}
